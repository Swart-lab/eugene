% Documentation of the MarkovProt sensor

\subsection{\texttt{Sensor.MarkovProt}}

\paragraph{Description}

This plugin injects coding/non coding likelihood as modeled by proteic
Markov models. These models are defined in a matrices file (located in
the directory specified by the \texttt{EuGene.PluginsDir} parameter)
whose name is indicated by the \texttt{MarkovProt.matname} parameter.
The order of the Markov model must be given in
\texttt{MarkovProt.maxorder} while the actual order to use is set by
\texttt{MarkovProt.order}.

The plugin is controlled by two further parameters:
\texttt{MarkovProt.minGC} and \texttt{MarkovProt.maxGC} which indicate
the GC scope of the matrices. If the GC\% of the sequence is out of
the scope, the plugin will give an equal null loglikelihood to all types
of regions.

The sensor is activated by setting the value TRUE for the parameter
\texttt{Sensor.MarkovProt.use} in the parameter file.

Here is an example of MarkovProt parameters definition.
\begin{Verbatim}[fontsize=\small]
MarkovProt.matname      SwP41.noFragm.mininfo1.order2.bin
MarkovProt.minGC        0
MarkovProt.maxGC        100
MarkovProt.maxorder     2
MarkovProt.order        2
Sensor.MarkovProt.use   TRUE    # Use MarkovProt sensor
Sensor.MarkovProt       1       # Sensor priority
\end{Verbatim}

\paragraph{Input files format}

No input files  needed beyond the markov matrix files.

\paragraph{Filtering input information}

No filtering.

\paragraph{Integration of information}

For coding tracks, assuming a uniform codon usage, the probability of
the coding tracks is decomposed as the product of choosing a codon and
then emitting the corresponding amino acid in the corresponding
phase. The logarithm of the probability is used for weighting.

For other tracks, a simple GC\% model is used to compute a background
probability. The logarithm of the probability is used for weighting.

\paragraph{Post analyse}

No post analyse.

\paragraph{Graph}

Same as in the MarkovIMM plugin.  


