% Documentation of the Splice WAM sensor

\subsubsection{\texttt{Sensor.SpliceWAM}}

\paragraph{Description}

The goal of the SpliceWAM sensor is to detect splice sites and to
give them a score reflecting the context accordance with given
models. A score is attributed at each potential acceptor and donor
AG/GT site according to Weight Array Method (see Zhang and Marr, {\em
Comput Appl Biosci.} 1993 Oct;9(5):499-509), or Weighted Array Matrix
models (Salzberg, {\em Comput Appl Biosci} 1997 Aug;13(4):365-76). A
WAM describes a consensus motif of a functional signal, and is
composed by one markovian model per each position of the motif. Here
the motifs are defined by the AG/GT (assumed to be present in all
splice sites) plus two flanking contexts (used by the WAM).  Globally,
the score of a motif is a function of the emission probabilities of
this motif given a true positive model and a false positive model.

The sensor is activated by setting the parameter
\texttt{Sensor.SpliceWAM.use} to TRUE.  The user have to specify in
the parameter file the base name (prefix) of the model files
(\texttt{SpliceWAM.donmodelfilename} and
\texttt{SpliceWAM.accmodelfilename}), the size of the context
(\texttt{SpliceWAM.NbNtBeforeGT, SpliceWAM.NbNtAfterGT} and
\texttt{SpliceWAM.NbNtBeforeAG, SpliceWAM.NbNtAfterAG}), the order of
the markovian models \texttt{SpliceWAM.MarkovianOrder} (the same for
each position of the motifs), and the scaling parameters
\texttt{SpliceWAM.DonScaleCoef*}, \texttt{SpliceWAM.DonScalePenalty*},
\texttt{SpliceWAM.AccScaleCoef*} and \texttt{SpliceWAM.ScalePenalty*}.

Here is an example of SpliceWAM parameters definition.
\begin{Verbatim}[fontsize=\small]
SpliceWAM.MarkovianOrder        1
SpliceWAM.donmodelfilename      WAM/WAM.ARA.DON.L9
SpliceWAM.NbNtBeforeGT          3
SpliceWAM.NbNtAfterGT           4
SpliceWAM.DonScaleCoef*         2.9004
SpliceWAM.DonScalePenalty*      -7.5877
SpliceWAM.accmodelfilename      WAM/WAM.ARA.ACC.L7
SpliceWAM.NbNtBeforeAG          2
SpliceWAM.NbNtAfterAG           3
SpliceWAM.AccScaleCoef*         2.9004
SpliceWAM.AccScalePenalty*      -7.5877
Sensor.SpliceWAM.use            TRUE               # Use SpliceWAM sensor
Sensor.SpliceWAM                1                  # Sensor priority
\end{Verbatim}

\paragraph{Input files format}

This SpliceWAM Sensor requires a true positive and a false positive
model file per motif position and for each type so sites. These files
have to be present in the path given by
\texttt{SpliceWAM.don\-modelfilename} and
\texttt{SpliceWAM.donmodelfilename} from the plugins directory (see
\texttt{EuGene.\-PluginsDir} parameter). These models can be generated
using \texttt{WAMbuilder.cc} (see
\texttt{eugene/src/\-SensorPlugins/0\_SensorTk/GetData/README}).  The
file name of a model is a concatenation of the base name (prefix)
specified in the parameter file, an extension (suffix) specified in
the \texttt{WAM.h} file (\texttt{.TP.} for true positive and
\texttt{.FP.}  for false positive), and a number between 00 and 99
indexing the position in the motif (restricting thus the motif length
to a maximum of 100 nt).

As an example, with the base name \texttt{WAM.ARA.DON.L9} (refering to
{\em A.thaliana} models of 9nt-length donor motif), one can found these files:
\begin{Verbatim}[fontsize=\small]
WAM.ARA.DON.L9.FP.00
WAM.ARA.DON.L9.FP.01
...
WAM.ARA.DON.L9.FP.07
WAM.ARA.DON.L9.FP.08
WAM.ARA.DON.L9.TP.00
WAM.ARA.DON.L9.TP.01
...
WAM.ARA.DON.L9.TP.08
\end{Verbatim}

These files are in binary form, each containing the properties of a
markovian model (see documentations of \texttt{WAMbuilder.cc} and
\texttt{markov.cc}).

\paragraph{Filtering input information}

Each binary model file is verified when loaded, checking if 3 expected
properties of its markovian model are verified: the order, the
alphabet size, and the total number of possible words (these 3 values
are automatically included during the models generation by
\texttt{WAMbuilder.cc}). This test is done in the loading file method
``\texttt{chargefichier}'' in markov.cc.

\paragraph{Integration of information}

At each AG/GT occuurence in the genomic sequence, a score is assigned
depending on the AG/GT flanking context. If there isn't enough
context, e.g. in the sequence extremities, nothing is done.  This
score is provided by a scaled sum of likelihood ratio, computed as
following.

Let be $P^t_i$ the emission probability of the nucleotid at position
$i$ in the motif according to the True Positive model, and $P^f_i$ the
emission probability of the nucleotid given by the False Positive
model. The score given by the WAM for the entire motif $M$ of length
$L$ is:
\[ S_M = \sum_{i=0}^{L} log\left(\frac{P^t_i}{P^f_i}\right) \]

This score is then scaled with the
\texttt{SpliceWAM.DonScaleCoef*}/\texttt{SpliceWAM.AccScaleCoef*} and
the
\texttt{SpliceWAM.DonScalePenalty*}/\texttt{SpliceWAM.AccScalePenalty*}
parameters, following this formula :
$$S_M . \texttt{SpliceWAMScaleCoef*} + \texttt{SpliceWAMScalePenalty*}$$

This rescaled score is finally integrated into the \EuGene\ graph on
the intron/exon transition edges at the corresponding positions. The
score applies only to the edge corresponding to the situation where
the signal is used. The edge corresponding to the situation where the
signal is not used is unchanged.

\paragraph{Post analyse}

No Post-Analyse.

\paragraph{Graph}

Vertical green/magenta lines (whose length is function of the score)
are plotted on the intron track on the corresponding strand for each
splice site occurrence whose score is higher than a defined
threshold. This threshold is defined in the \texttt{SpliceWAM.cc} file
as -\texttt{plotscoreincrease}.
