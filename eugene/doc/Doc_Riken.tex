% Documentation of the Riken sensor

\subsection{\texttt{Sensor.Riken}}

Todo.

\paragraph{Description}

A file with extension \texttt{.riken} is read.  Each line must contain
the positions of the extremities of the match of the 5' EST then the
name of the 5' EST the same thing for the 3'EST then the name of the
clone.

The sensor is activated by either :
\begin{itemize}
\item the \texttt{-R} argument \index{CmdFlags}{[Riken activation] R}
\item the value TRUE for the parameter \texttt{Sensor.Riken.use} in the
  parameter file.
\end{itemize}
Here is an example of Riken parameters definition :
\begin{Verbatim}[fontsize=\small]
Riken.StrandRespect             0
Riken.Min_est_diff              100
Riken.Max_overlap               60
Riken.Max_riken_length          60000
Riken.Max_riken_est_length      3000
Riken.Min_riken_length          120 
Riken.Min_riken_est_length      10
Riken.RAFLPenalty*              -120
Sensor.Riken.use                TRUE     # Use Riken sensor
Sensor.Riken                    7        # Sensor priority
\end{Verbatim}

\paragraph{Input files format}


\paragraph{Filtering input information}


\paragraph{Integration of information}


\paragraph{Post analyse}


\paragraph{Graph}






