% Documentation of the Riken sensor

\subsection{\texttt{Sensor.Riken}}

\paragraph{Description}

The plugin allows to exploit 5'/3' EST extracted from the extremities
of full-length cDNA. This type of data was produced by the Riken
institute for Arabidopsis thaliana. By mapping such EST to the genomic
sequence, it is possible to know the positions where a gene
(transcript) must start (5' side) and stop (3' side). The plugin
assumes that this mapping has been done and that the coordinates of
the extremities of the 5' and 3' EST of full-length clones have been
determined before hand.

The sensor is activated by either :
\begin{itemize}
\item the \texttt{-R} argument \index{CmdFlags}{[Riken activation] R}
\item the value TRUE for the parameter \texttt{Sensor.Riken.use} in the
  parameter file.
\end{itemize}

The plugin is controled by several parameters, most of which control
sanity checks (see below). The \texttt{Riken.RAFLPenalty*} parameter
controls the amount of penalty used to force \EuGene\ to predict a
gene on a region defined by a valid EST pair.

Here is an example of Riken parameters definition :
\begin{Verbatim}[fontsize=\small]
Riken.Min_est_diff              100
Riken.Max_overlap               60
Riken.Max_riken_length          60000
Riken.Max_riken_est_length      3000
Riken.Min_riken_length          120 
Riken.Min_riken_est_length      10
Riken.StrandRespect             0
Riken.RAFLPenalty*              -120
Sensor.Riken.use                TRUE     # Use Riken sensor
Sensor.Riken                    7        # Sensor priority
\end{Verbatim}

\paragraph{Input files format}

A file with extension \texttt{.riken} is read.  Each line must contain
the positions of the extremities of the match of the 5' EST then the
name of the 5' EST, the same thing for the 3'EST and finally the name
of the clone.

Here is an exert of a typical \texttt{.riken} file:
\begin{Verbatim}[fontsize=\small]
417757  418379  AV826766        418902  419330  AV796216        0907A18
341382  342036  AU235278        340748  341549  AU225941        1201K23
40318   40969   AV821185        38800   39323   AV781490        0208M10
309757  310341  AV830906        308043  308392  AV813791        0980B11
387624  388227  AU236666        387383  387834  AU227623        1514C21
148345  148909  AV822910        147090  147960  AV783778        0513A17
\end{Verbatim}

\paragraph{Filtering input information}

The plugin uses several parameters that control sanity
checks on the input data.
\begin{itemize}
\item the \texttt{Riken.Max\_riken\_length} parameter controls the
  maximum length for a transcript. If an EST pair defines a transcript
  with a length of more than this number of base pairs, then it is
  ignored. A typical value is 60kb (for \textit{Arabidopsis
    thaliana}).
\item the \texttt{Riken.Min\_riken\_length} parameter controls the
  minimum length for a transcript. If an EST pair defines a transcript
  with a length lower than this number of base pairs, then it is
  ignored. A typical value is 120b (for \textit{Arabidopsis
    thaliana}).
\item the \texttt{Riken.Max\_riken\_est\_length} parameter controls the
  maximum length of the genomic sequence matching one EST. If either
  the 5' or the 3' EST exceed this length, then the EST pair is
  rejected. A typical value is 3kb (for \textit{Arabidopsis
    thaliana}).
\item the \texttt{Riken.Min\_riken\_est\_length} parameter controls the
  minimum length of the genomic sequence matching one EST.  If either
  the 5' or the 3' EST are below this length, then the EST pair is
  rejected. A typical value is 10 bp (for \textit{Arabidopsis
    thaliana}).
\item the \texttt{Riken.StrandRespect} parameter controls whether the
  5'/3' information available for the EST is taken into account or
  ignored. If this parameter is set to \texttt{0}, then the
  information is ignored and the prediction of a gene is ``forced'' in
  the region but with no constraint on the strand.  Otherwise, and if
  the 5'/3' EST pair is separated enough to decide the strand, then
  the prediction of a gene is forced on the strand detected.
\item the \texttt{Riken.Min\_est\_diff} parameter controls the minimum
  distance of separation between the 5' and 3' EST (computed as the
  sum of the distances of the left and right extremities of the two
  genomic sequences mapping the 2 EST) that is sufficient to deduce
  the strand of the gene. A typical value is 100 bp (for
  \textit{Arabidopsis thaliana}).
\item the \texttt{Riken.Max\_overlap} parameter controls how
  information on ``overlapping'' regions is handled. If two EST pairs
  define transcribed regions with a large overlap (larger than the
  parameter value), then it is likely that they refer to the same gene
  (as far as they are detected as being on the same strand). In this
  case, the two EST pairs are taken as one (merged by taking the
  leftmost extremity as the new left extremity and the rightmost as
  the right extremity). If the two overlapping regions are not on the
  same strand, then they are considered as inconsistent and the
  orientation is forgotten.
  
  If the two regions have a small overlap (lower than the parameter
  value), then it is likely that there are 2 different genes with
  overlapping UTR. Because \EuGene\ cannot predict overlapping UTR,
  then the extremities are modified so that they do not overlap
  anymore.
\end{itemize}

\paragraph{Integration of information}

Basically, when a genomic region is validated, the plugin forces
\EuGene\ to predict one single gene in the region. This is done by
penalizing all tracks but the intergenic track just before and after
the gene extremities and by penalizing the intergenic track on the
genomic region itself. If the strand is also considered as detected,
then all tracks on the other strand are also penalized.  Although an
infinite (eg. \texttt{-1e999} in the current double format) penalty
would seem more appropriate, we advocate for a strong finite penalty
to avoid stupid uselss predictions in case of data inconsistency.

\paragraph{Post analyse}

No post analyze.

\paragraph{Graph}

The Riken information is plotted on the output graph as two small
corners delimiting the region on the intergenic track. The corners are
colored differently according to the strand detected for the
transcribed region.





