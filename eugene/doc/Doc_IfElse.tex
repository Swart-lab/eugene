% Documentation of the IfElse sensor

\subsection{\texttt{Sensor.IfElse}}

\paragraph{Description}

This plugin is used to combine the predictions of two existing
plugins. It listen to a first plugin. For each possible predictable
item, if this plugin predicts something then this prediction is used.
If the plugin does not predict anything, then the output of the second
plugin is used.
 
The plugin needs only two parameters to be informed:
\texttt{IfElse.SensorIf} and \texttt{IfElse.SensorElse} which indicate
the names of the two slave plugins. The two slave plugins will be
loaded with an instance number equal to one plus the instance number
of the IfElse sensor itself (allowing for nested IfElse).

Here is an example of IfElse parameters definition which uses the NG2
Sensor if it predicts something or else the SplicePredictor Sensor.
\begin{Verbatim}[fontsize=\small]
IfElse.SensorIf         NG2
IfElse.SensorElse       SPred
Sensor.IfElse.use       TRUE    # Use IfElse sensor
Sensor.IfElse           1       # Sensor priority
\end{Verbatim}

In this case, since the IfElse is loaded as a first plugin (instance
0), the two slave plugins will be instaciated as instance number
one. The parameters for the 2 plugins must therefore be suffixed by
\texttt{[1]}.

\paragraph{Input files format}

No input files  needed beyond those used by the slave sensors.

\paragraph{Filtering input information}

No filtering.

\paragraph{Integration of information}

The ``If'' plugin is called. For each of the possible information type
(signal and contents), if nothing is predicted by it, the prediction
of the second plugin is used instead.

\paragraph{Post analyse}

No post analyse beyond the post analyze in the slave plugins.

\paragraph{Graph}

Nothing beyond the plotting in the slave plugins.


