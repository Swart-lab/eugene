% Documentation of the Est sensor

\subsubsection{\texttt{Sensor.Est}}
\label{plugest}

\paragraph{Description}

This sensor is intended to take into account information from aligned
transcribed sequences, both complete cDNA and EST. The existence of a
hit (resp. gap) in the spliced alignment will influence intergenic,
exonic and intronic state costs by penalizing states that are
incompatible with the alignment. The spliced alignments must be
performed beforehand using a spliced aligner such as \texttt{sim4} or
\texttt{spidey}. The output of these aligners must be converted in the
adequate format (see below).  To also take into account information
from ncRNA hits, set \texttt{Est.mRNAOnly} to 0.

The sensor is activated by either:
\begin{itemize}
\item the \texttt{-d} argument \index{CmdFlags}{[EST activation]
    d}.
\item a value higher than 0 for the parameter \texttt{Sensor.Est.use} in
  the parameter file.
\end{itemize}

The behavior of the plugin is controlled by the following parameters:
\begin{itemize}
\item \texttt{Est.estP*} indicates the penalty for violating a transcribed
  evidence.
\item \texttt{Est.SpliceBoost*} is a bonus used to enhance the score 
  of predicted splice sites which appear at a hit/gap border of aligned 
  spliced ESTs.
\item \texttt{Est.estM} gives the amount of ``fuzzyness'' allowed in
  interpreting a hit/gap border. The nucleotides which are less than
  \texttt{Est.estM} nucleotides away from this border are considered
  as neither in a hit or a gap.
\item \texttt{Est.utrP*} is a penalty introduced to try to limit the
  extension of UTR beyond the frontier of transcribed evidence when
  there is some. For a defined length, the adequate UTR states that
  precede or follow an uninterrupted stretch of transcribed evidence
  will be penalized by the logarithm of this parameter.
\item \texttt{Est.utrM} gives the number of UTR nucleotides that will
  be penalized using the previous penalty on the border of transcribed
  evidence.
\item \texttt{Est.StrongDonor} gives a threshold $T$ on donor strength
  inside intronless EST. If a given Donor with scores $a|b$ such that
  $\log(\frac{a}{b}) >T$ then the intronless EST is rejected because
  it goes through a very strong donor site.
\item \texttt{Est.MinDangling} gives the minimum length of the first
  and last match region in a genomic spliced alignment. If the length
  is below this, then it is assumed to be a false match and it is
  ignored.
\item \texttt{Est.MaxIntron} gives the maximum length for the first
  and last gap (representing introns) in a spliced alignment. If the
  length exceeds this maximum, then the corresponding regions are
  ignored in the alignment (the long gap and the dangling hit).
\item \texttt{Est.MaxInternalIntron} gives the maximum length for any
  gap (representing an intron) in a spliced alignment. If the length
  exceeds this maximum, then the corresponding gap is ignored as
  a probable bad spliced alignment or chimeric data.
\item \texttt{Est.mRNAOnly} In the default mode (1), the plugin reads 
transcribed sequences which are mRNA fragments or complete cDNA, 
so the plugin always penalizes ncRNA states. Set this parameter to 0 
to read non coding RNA fragments too: in this case, only ncRNA tracks 
which are incompatible with the alignment will be penalized.

\end{itemize}

The sensor is also capable of a postprocessing analyse described below
and activated either by the \texttt{-E} argument or by setting the value
1 or 2 for the parameter \texttt{Est.PostProcess} in the parameter
file.

The \texttt{Sensor.Est} is loaded last because it has the highest priority.
This is important since the sensor actually uses the information
provided by other sensors (splice site prediction sensors) that then have to be loaded before.

Here is an example of Est parameters definition.
\begin{Verbatim}[fontsize=\small]
Est.PostProcess[0]       0      # 0 1 OR 2
Est.PPNumber[0]          5      # For PostProcess = "2"
Est.estP*[0]             -0.4
Est.estM[0]              6
Est.utrP*[0]             0.35
Est.utrM[0]              5
Est.SpliceBoost*[0]      0.0
Est.StrongDonor[0]       0.95
Est.MinDangling[0]       10
Est.MaxIntron[0]         15000
Est.MaxInternalIntron[0] 15000
Est.mRNAOnly[0]          1
Est.FileExtension[0]     .est
Sensor.Est.use           1      # Use EST sensor
Sensor.Est               20     # Sensor priority: the highest one

\end{Verbatim}


\paragraph{Native input files format}

The plugin reads information from the file whose name is the concatenation 
of the sequence file name and the \texttt{Est.FileExtension} parameter. 
Each line of the input file is composed of 8 fields:


\texttt{<$b^S$> <$e^S$> <$s^S$> <$x$> <$b^S$> <est name> <$bq^S$> <$eq^S$>}

where:
\begin{itemize}
\item $b^S$ and $e^S$ are the begin and the end of the similarity S on the genomic sequence,

\item $s^S$ is the score of the similarity S (number of identical bases)
\item $x$ is unused for now
\item $b^S$ is the strand where the similarity occurs (forward = 0,
  reverse = 1). This information is not used anymore by the plugin
  which decides the strand of similarity by itself if there is anough
  information.
\item \texttt{<est name>} is the name of the EST/cDNA sequence. Each
  sequence MUST have a unique name.
\item <$bq^S$> <$eq^S$> are the begin and the end of the similarity S
  on the query (EST/cDNA) sequence.
\end{itemize}
The lines in the file must be ordered by sequence name first (all the
hits of a given EST are put together) and by increasing
\texttt{<$bq^S$> <$eq^S$>}. 

Here is an example for the \texttt{SYNO\_ARATH.tfa.est} file:
\begin{Verbatim}[fontsize=\small]
    32    421 1844 0 0 ATAJ644    1  390
   514    582 1844 0 0 ATAJ644  391  459
   699    809 1844 0 0 ATAJ644  460  570
   914   1018 1844 0 0 ATAJ644  571  675
  1271   1408 1844 0 0 ATAJ644  676  813
  1522   1602 1844 0 0 ATAJ644  814  894
  1694   1771 1844 0 0 ATAJ644  895  972
  1853   1921 1844 0 0 ATAJ644  973 1041
  2014   2088 1844 0 0 ATAJ644 1042 1116
  2181   2264 1844 0 0 ATAJ644 1117 1200
  2360   2446 1844 0 0 ATAJ644 1201 1287
  2712   2882 1844 0 0 ATAJ644 1288 1458
  2966   3092 1844 0 0 ATAJ644 1459 1585
  3189   3447 1844 0 0 ATAJ644 1586 1844
    32    375 347 0 0 AT00622    1  347
  3071   3092 256 0 1 AI994358    1   22
  3189   3421 256 0 1 AI994358   23  256
\end{Verbatim}

All the hits of the ATAJ644 are clustered together and sorted with
increasing \texttt{<$bq^S$> <$eq^S$> }. In practice, this file can be
directly constructed from an EST/cDNA bank and the sequence using a
modified version of \texttt{sim4}. This version outputs splices
alignements in the correct format (using the flag \texttt{A=6}) and
only outputs hits with a coverage of more than 80\%\ and with a
similarity either than 90\%.

A patch file \texttt{sim4.patch} is available in the plugin source
directory as well as an awk script that put a FASTA sequence bank in a
pure sim4 format (no upper case, no degenerated code). This seems
useless on recent sim4 versions.

\paragraph{Gff3 input file format}

The gff3 input mode is activated by setting the value \texttt{GFF3}
for the parameter \texttt{Est.format} in the parameter file. The
plugin reads its information from the file whose name is the concatenation
of the sequence file name, the \texttt{Est.FileExtension} parameter 
and the \texttt{.gff3} extension. Each line is as follows:


\texttt{<sequence name> <program> <sofa feature> <$b^S$> <$e^S$> <$v^S$> 
<strand> <.> \\
ID=...;Target=<est name> <$bp^S$> <$ep^S$> <est strand>;score\_hit=<$s^S$>}

Accepted features (third column):
\begin{itemize}
\item  SO:0000668 or EST\_match
\end{itemize}


Gff3 attributes specifications:\\
Required attributes:
	\begin{itemize}
	\item ID
	\item Target
	\end{itemize}

Optional attributes:
	\begin{itemize}
%	\item frame\_hit represent the blast frame. It can be omitted but if specified, its consistency with the hit coordinates will be cheked and the hit will be rejected if the frame is not consistent.
	\item score\_hit :  normalized score.
	\end{itemize}


Here an extract of : seq14ac002535g4g5.tfa.est.gff3
\begin{Verbatim}[fontsize=\tiny]
seq14	gth	EST_match	3261	4698	.	-	.	ID=EST_match:seq14.1;
seq14	gth	EST_match	3261	3332	0	-	.	ID=EST_match:seq14.1.1;Target=AV537418 1 72 +;score_hit=636;Parent=EST_match:seq14.1;
seq14	gth	EST_match	3411	3494	0	-	.	ID=EST_match:seq14.1.2;Target=AV537418 73 156 +;score_hit=636;Parent=EST_match:seq14.1;
seq14	gth	EST_match	3583	3657	0	-	.	ID=EST_match:seq14.1.3;Target=AV537418 157 231 +;score_hit=636;Parent=EST_match:seq14.1;
seq14	gth	EST_match	4209	4301	0	-	.	ID=EST_match:seq14.1.4;Target=AV537418 232 324 +;score_hit=636;Parent=EST_match:seq14.1;
seq14	gth	EST_match	4387	4698	0	-	.	ID=EST_match:seq14.1.5;Target=AV537418 325 636 +;score_hit=636;Parent=EST_match:seq14.1;
\end{Verbatim}


\paragraph{Filtering input information}

The EST information goes through a complex filtering process. First
all hits are loaded. Successive hits of a same sequence are considered
as a single alignment.  For every spliced alignement, the plugin
checks if a splice site of the correct type as been predicted near the
border of each gap (less than \texttt{Est.estM} bases aways of the
border). This is checked on each strand. If a strand does not contain
the necessary splice sites, then it is considered as impossible. If
neither strand contains adequate splice sites, the sequence is
discarded (filtered).

All remaining alignments are sortered by 1) decreasing number of
detected gaps then by 2) length (this tends to put cDNA or spliced EST
alignments first) and 3) by the alphabetical order of the sequence
names (to avoid sorting ambiguities). Any sequence that is
inconsistent with previous sequence in this order (in the sense that
they indicate that a given nucleotide is part of an intron, resp.
exon, while a previous sequence indicates that the nucleotide is part
of an exon (resp. intron) is discarded (filtered).

Any unspliced sequence that crosses a donor site that is predicted
with a sufficiently strong confidence (See \texttt{Est.StrongDonor*})
is filtered out.

\paragraph{Integration of information}

Using filtered EST/cDNA sequence that are all consistent, every
nucleotide can either be located:
\begin{description}
\item[Hit] inside a matching segment that can occur on the forward strand
  on the reverse strand or both (as identified during filtering).
\item[Gap] or inside a gap segment that can occur on the forward strand
  on the reverse strand or both (as identified during filtering).
\item[None] or otherwise outside of any existing hit or gap (or less
  than \texttt{Est.estM} bases aways of the border of such a segment)
\end{description}

For a Hit: all intronic and intergenic tracks as well as UTR and
exonic tracks on a strand incompatible with the hit are penalized with
\texttt{Est.estP*}.

For a Gap: all exonic and intergenic tracks as well as UTR and
intronic tracks on a strand incompatible with the hit are penalized with
\texttt{Est.estP*}.

When no information exists (None), the UTR tracks are penalized by
\texttt{Est.utrP*} if and only if there is a Hit evidence at less than
\texttt{Est.utrM} bases away.

\paragraph{Post analysis}

The correspondence between transcribed sequence information and
prediction is analyzed if the \texttt{-E} argument
\index{CmdFlags}{[EST postprocessing activation] E} is activated or if
the \texttt{Est.PostProcess} parameter value is \texttt{1} or \texttt{2}.

\begin{itemize}
 \item \texttt{Est.PostProcess = 1}: EST/cDNA are compared to the prediction

For each predicted transcript (from 5'UTR to 3'UTR), each available
EST/cDNA sequence that overlaps the transcript is compared to the
prediction. The transcribed sequence is then classified as:
\begin{itemize}
\item Filtered (if it was filtered in the initial filtering process)
\item Inconsistent (if it is incompatible with the prediction)
\item Full transcript Support (if it is completely consistent with the
  predicted transcript on all the predicted transcript length)
\item Full Coding Support (if it is completely consistent with the
  predicted CDS on all the predicted CDS length, from start to stop
  codon)
\item Support (if it is otherwise consistent with the prediction)
\end{itemize}
Finally, for each predicted transcript (from 5'UTR to 3'UTR) and
predicted CDS (from start to stop), the number of predicted
nucleotides that are supported by existing transcripts (filtered or
not) is reported.

 \item \texttt{Est.PostProcess = 2}: the prediction is compared to the EST/cDNA

For each predicted coding exon, the percentage of nucleotides
supported by a transcribed sequence is displayed. This count includes
all EST/cDNA, including those filtered. 
Overlapped transcribed sequences are displayed in the last column.
The \texttt{Est.PPNumber} parameter is the maximum number of displayed 
transcribed sequences per exon.
\end{itemize}


\paragraph{Graph}

Non filtered transcribed evidence are plotted as blue horizontal
blocks (hits) separated by thin blue lines (gap) on the intronic
tracks. If the strand of the transcription has been identified during
filtering, the blocks and lines occurr only on the corresponding IR or
IF track.

Filtered sequences are plotted as gray blocks and lines just above
(below) the forward (reverse) intronic tracks.

