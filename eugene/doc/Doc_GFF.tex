% Documentation of the GFF sensor

\subsection{\texttt{Sensor.GFF}}

\paragraph{Description}

The GFF sensor allows to add to the graphical representation an
annotation provided in a GFF format. Note that the provided GFF
annotation could be an \EuGenie prediction given in GFF format
(obtained using the $-pg$ argument). This could allow to visualise two
predictions on the same graph.

For a sequence, the plugin reads the annotation from one file whose
name is derived from the sequence name by adding the \texttt{.gff}
suffix.  The sensor is activated by either :
\begin{itemize}
\item the \texttt{-G} argument \index{CmdFlags}{[GFF activation] G}
\item the value TRUE for the parameter \texttt{Sensor.GFF.use} in the
  parameter file.
\end{itemize}
Here is an example of GFF parameters definition :
\begin{Verbatim}[fontsize=\small]
Sensor.GFF.use  TRUE      # Use GFF sensor
Sensor.GFF      10        # Sensor priority
\end{Verbatim}

\paragraph{Input files format}

The files \texttt{.gff} describe an annotation for a sequence. The format of
a line is : \texttt{<seqname> <source> <feature> <start> <end> <score>
  <strand> <frame>}. Seqname, source and score fields are ignored.

Example:
\begin{Verbatim}[fontsize=\small]
seqName  EuGene  Utr5    1       199     0       -       .
seqName  EuGene  Utr5    340     359     0       +       .
seqName  EuGene  Init    360     393     0       +       2
seqName  EuGene  Intr    596     732     0       +       0
seqName  EuGene  Intr    830     876     0       +       1
seqName  EuGene  Intr    961     1286    0       +       1
seqName  EuGene  Intr    1396    1478    0       +       2
seqName  EuGene  Intr    1573    1648    0       +       0
seqName  EuGene  Intr    1757    1818    0       +       0
seqName  EuGene  Intr    1962    2057    0       +       2
seqName  EuGene  Intr    2145    2306    0       +       2
seqName  EuGene  Term    2491    2607    0       +       0
seqName  EuGene  Utr3    2608    2626    0       +       .
\end{Verbatim}
Note: only exons are plotted, this file is parsing by the frame field
(no `.' in the frame field).

\paragraph{Filtering input information}

No filter.

\paragraph{Integration of information}

This sensor does not affect prediction.

\paragraph{Post analyse}

No post analyse.

\paragraph{Graph}

Orange horizontal lines are plotted on the exon tracks.

