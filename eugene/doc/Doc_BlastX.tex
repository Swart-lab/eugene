% Documentation of the BlastX sensor

\subsection{\texttt{Sensor.BlastX}}

\paragraph{Description}

The BlastX sensor allows to exploit similarities with homologous
proteins. The similarities influence exon and intron detection.
Similarities from several databases can be exploited. Usually 3
databases are used: SwissProt, PIR and TrEMBL.

A label $i$ (that could vary from 0 to 9) is assigned at each
considered database. iles describing a collection of similarities with
a sequence have an extension .blast$<$i$>$ (\texttt{.blast0,...,.blast9}).

The user has to specify the list of labels to consider, the
confidence accorded to each, the minimum length of an intron and a
number of amino acids involved in intron incitation.  The sensor is
activated by either:
\begin{itemize}
\item the \texttt{-b} argument \index{CmdFlags}{[BlastX activation] b}
  that allows to specify the labels to consider, for example
  \texttt{-b092} to use the levels 0, 9, 2 (files \texttt{.blast0,
    .blast9, .blast2}),
\item the value TRUE for the parameter \texttt{Sensor.BlastX.use} in
  the parameter file and the labels to consider in the
  \texttt{.BlastX.levels} parameter.
\end{itemize}
The confidence in analyzes have to be specified in the parameter file
giving values to the parameters \texttt{BlastX.level}$<$i$>$.  The
minimum length of an intron is defined in the \texttt{EuGene.minIn}
parameter.  A number of amino acids defined in the
\texttt{BlastX.blastxM} parameter that allows to define if 2
similarities are near (see the paragraph Integration of information).
Finally the \texttt{BlastX.postProcess} parameter (when set to TRUE)
allows to request to analyse how BlastX information are integrated in
the final prediction.

Here is an example of BlastX parameters definition.
\begin{Verbatim}[fontsize=\small]
Sensor.BlastX.use  TRUE     # Use BlastX sensor
Sensor.BlastX      2        # Sensor priority
BlastX.postProcess TRUE     # analyse prediction accorded to BlastX information
BlastX.levels      012      # use levels 0, 1, and 2
BlastX.level0*     0.2      # confidence in the level 0
BlastX.level1*     0.0      # confidence in the level 1
BlastX.level2*     0.0      # confidence in the level 2
BlastX.blastxM*    10       # nb of amino acids implicated in intron incitation
EuGene.minIn       50       # minimum length of intron
\end{Verbatim}

\paragraph{Input files format}

The files \texttt{.blast}$<$i$>$ describe a collection of similarities
sorted by protein and by position on the sequence. One similarity S is
described per line.

The format of a line is:

\texttt{<$b^S$> <$e^S$> <$s^S$> <$v^S$> <$p^S$> <protein name> <$bp^S$> <$ep^S$>}

where:
\begin{itemize}
\item $b^S$ and $e^S$ are the begin and the end of the similarity S on the sequence,
\item $s^S$ is the score of the similarity S,
\item $v^S$ is the e-value given by BlastX and ignored by \EuGenie,
\item $p^S$ is the phase: +1, +2, +3, -1, -2, -3,
item $bp^S$ and $ep^S$are the begin and the end of the similarity S on the protein.
\end{itemize}

Here is an extract from \texttt{SYNO\_ARATH.fasta.blast0}:
\begin{Verbatim}[fontsize=\small]
2820 2861 36 3e-08 +3  sp_O07683_SYD_HALSA; 335 348
2972 3088 41 3e-08 +2  sp_O07683_SYD_HALSA; 359 397
3185 3298 113 3e-08 +2  sp_O07683_SYD_HALSA; 398 435
353 418 45 2e-13 +2  sp_O24822_SYD_HALVO; 13 34
1850 1915 67 2e-13 +2  sp_O24822_SYD_HALVO; 202 223
2775 2858 72 2e-13 +3  sp_O24822_SYD_HALVO; 318 345
3191 3280 104 2e-13 +2  sp_O24822_SYD_HALVO; 397 426
353 418 51 7e-12 +2  sp_O26328_SYD_METTH; 21 42
1271 1414 70 7e-12 +2  sp_O26328_SYD_METTH; 141 188
1850 1954 62 7e-12 +2  sp_O26328_SYD_METTH; 210 244
3191 3280 93 7e-12 +2  sp_O26328_SYD_METTH; 401 430
\end{Verbatim}
These files can be obtained directly from the output BlastX files by
parsing them with the \texttt{blast\_parser.pl} script.
The BlastX is launched with the command:
\begin{Verbatim}
  blastall -p blastx -d DATABASE_MULTIFASTA_PROTEIC_FILE -g F -F T -b
  500000 -v 500000 -e 1e-6 -i QUERY_GENOMIC_SEQUENCE_FASTA >
  TEMPORY_BLAST_RESULT_FILE
\end{Verbatim}
and the final \texttt{.blast}$<i>$ files are obtained with:
\begin{Verbatim}
  blast_parser.pl TEMPORY_BLAST_RESULT_FILE | sort -n -k 1,1 | sort -s
  -k 6,6 > QUERY_GENOMIC_SEQUENCE_FASTA.blast0
\end{Verbatim}

For more explanation, see the README file in the directory
\texttt{eugene/SensorPlugins/BlastX/GetData}.

\paragraph{Filtering input information}

Similarities with a length higher than 15,000 nucleodites are
rejected. A message ``Similarity of extreme length rejected'' is
printed to alert the user.

\paragraph{Integration of information}

The procedure consists first, in computing information at the
nucleotide level and second, in weigthing the graph used by \EuGenie.

A/ Computing information at the nucleotide level\\
A-1/ Extracting information

Each similarity S is considered, one after the other.  A set of 3
variables is computed for nucleotide in position $i$. The variables
are:
\begin{itemize}
\item $s_i$ the score of the nucleotide at position i
\item $c_i$ the confidence in $s_i$,
\item $p_i$ the phase of $s_i$ : +1, +2, +3, -1, -2, -3 for exon and 0 for intron,
\end{itemize}
Let $l^S$ be the length of the similarity in nucleotide.
\[l^S = (ep^S -bp^S -1)*3\]

\underline{Valuation for exon position} \\

\begin{itemize}
\item from $i = b^S$ to  $i = e^S$
  \begin{itemize}
  \item $s_i = s^S / l^S$
  \item $c_i = c^S$
  \item $p_i = p^S$
  \end{itemize}
\end{itemize}

\underline{Valuation for intron position}\\

Intron is only possible if:
\begin{itemize}
\item the similarities before or after in the same file are on the
  same protein, strand and near. That is have the same protein name,
  the same sense and have a maximum distance or overlap of
  \texttt{BlastX.blastxM} amino acids.
\item the distance in nucleotide on the sequence is upper than
  \texttt{EuGene.minIn}.
\end{itemize}
Considering a similarity S with 2 similarities before and after in
accordance with these conditions, an intron is incitated on a small
length of \texttt{EuGene.minIn}/2 on both sides of S.
\begin{itemize}
\item from $i = b^S -$ EuGene.minIn$/2$ to $i = b^S$ 
\item from $i = e^S$ to $i = e^S +$ EuGene.minIn$/2$ 
\item the following values are given at each position:
  \begin{itemize}
  \item $s_i = s^S / l^S$ 
  \item $c_i = c^S$ 
  \item $p_i = 0$
  \end{itemize}
\end{itemize}

A-2/ Combining extracted information\\

When all the similarities have been handled, if a position has several
set of variables, the set with the highest confidence is kept. In
case of egal confidence, the set with the higher score is kept.

B/ Weigthing the graph\\

For each $i$ with a set of variables:
\begin{itemize}
\item if $p_i$ codes for exon then $s_i .c_i$ is added to the content
  signal of nucleotide i in the corresponding exon phase (a track
  between 0 and 5),
\item if $p_i$ codes for intron then $s_i .c_i$ is added to the intron
  content signals of nucleotide i (tracks 6 and 7),
\end{itemize}

Note: in fact, instead of rewarding the correct track (like described
here), all the tracks except the according one(s) are penalized, with
a penalty equal to $-|s_i .c_i|$.

\paragraph{Post analyse}

The correspondance between BlastX information and prediction is
analyzed if the \texttt{-B} flasg \index{CmdFlags}{[BlastX
  postprocessing activation] B} is provided or if the
\texttt{BlastX.PostProcess} parameter is set to \texttt{TRUE}.

For each predicted CDS, from the start codon to the stop, the
percentage of nucleotides supported by a proteic similarity is
displayed.

\paragraph{Graph}

Grey horizontal lines are ploted on the exon tracks for only the 3
first levels to consider (dark grey for the first, grey for the
second, and light grey for the third).

