% Documentation of the  Repeat sensor

\subsubsection{\texttt{Sensor.Repeat}}

\paragraph{Description}

The plugin allows to exploit the output of repeated sequences detector
such as RepeatMasker by penalizing exonic, inronic or UTR states when
repeats are detected.

The sensor is activated by either :
\begin{itemize}
\item the \texttt{-r} argument \index{CmdFlags}{[Repeat activation] r}
\item the value 1 for the parameter \texttt{Sensor.Repeat.use} in the
  parameter file.
\end{itemize}
The penalties used when a repeat exists are \texttt{Repeat.IntronPenalty*},
\texttt{Repeat.ExonPenalty*} and \texttt{Repeat.UTRPenalty*}
respectively.

Here is an example of Repeat parameters definition.
\begin{Verbatim}[fontsize=\small]
Repeat.UTRPenalty*      0.0
Repeat.IntronPenalty*   0.1
Repeat.ExonPenalty*     1.0
Sensor.Repeat.use       1    # Use Repeat sensor
Sensor.Repeat           1       # Sensor priority
\end{Verbatim}

\paragraph{Native input files format}

The file with a \texttt{.ig} suffix is needed. Each line of the file
contains the beginning and the end of a region detected as a
repeat. The positions must be sorted in increasing positions. Such a
file can be obtained by eg. reformatting RepeatMasker output.

Here is an extract from a typical \texttt{.ig} file:
\begin{Verbatim}[fontsize=\small]
4800    5006
22494   22758
22703   22772
22841   23017
22929   23017
29433   29703
[...]
\end{Verbatim}

\paragraph{Gff3 input file format}

The gff3 input mode is activated by setting the value \texttt{GFF3}
for the parameter \texttt{Repeat.format} in the parameter file.  The
plugin reads the file which name is derived from the sequence name by
adding the \texttt{.ig.gff3} extension.

Accepted features (third column):\\
\begin{itemize}
\item  SO:0000657 or repeat\_region
\end{itemize}
If the feature used isn't one of those, the line will be rejected. 
Here an extract of \texttt{seq14ac002535g4g5.tfa.ig.gff3}.
\begin{Verbatim}[fontsize=\tiny]
seq14	RepeatMasker	repeat_region	1	100	.	.	.	ID=repeat_region:seq14.1;
\end{Verbatim}

\paragraph{Filtering input information}

No filtering.

\paragraph{Integration of information}

For exonic, intronic and UTR tracks, all positions that occur in a
repeat interval as reported in the \texttt{.ig} file are penalized
using the corresponding \texttt{Repeat.IntronPenalty*},
\texttt{Repeat.ExonPenalty*} and \texttt{Repeat.UTRPenalty*} penalties.

\paragraph{Post analyse}

No post analyse.

\paragraph{Graph}

Repeat intervals are visualized as grey blocks in the intergenic
track.




