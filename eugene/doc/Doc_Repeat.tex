% Documentation of the  Repeat sensor

\subsection{\texttt{Sensor.Repeat}}

\paragraph{Description}

The plugin allows to exploit the output of repeated sequences detector
such as RepeatMasker by penalizing exonic, inronic or UTR states when
repeats are detected.

The sensor is activated by either :
\begin{itemize}
\item the \texttt{-r} argument \index{CmdFlags}{[Repeat activation] r}
\item the value TRUE for the parameter \texttt{Sensor.Repeat.use} in the
  parameter file.
\end{itemize}
The penalties used when a repeat exists are \texttt{Repeat.IntronPenalty},
\texttt{Repeat.ExonPenalty} and \texttt{Repeat.UTRPenalty}
respectively.

Here is an example of Repeat parameters definition.
\begin{Verbatim}[fontsize=\small]
Repeat.UTRPenalty*      0.0
Repeat.IntronPenalty*   0.1
Repeat.ExonPenalty*     1.0
Sensor.Repeat.use       TRUE    # Use Repeat sensor
Sensor.Repeat           8       # Sensor priority
\end{Verbatim}

\paragraph{Input files format}

The file with a \texttt{.ig} suffix is needed. Each line of the file
contains the beginning and the end of aq region detected as a
repeat. The positions must be sorted in increasing positions. Such a
file can be obtained by eg. reformatting RepeatMasker output.

Here is an extract from a typical \texttt{.ig} file:
\begin{Verbatim}[fontsize=\small]
4800    5006
22494   22758
22703   22772
22841   23017
22929   23017
29433   29703
[...]
\end{Verbatim}

\paragraph{Filtering input information}

No filtering.

\paragraph{Integration of information}

For exonic, intronic and UTR tracks, all positions that occur in a
repeat interval as reported in the \texttt{.ig} file are penalized
suing the corresponding \texttt{Repeat.IntronPenalty},
\texttt{Repeat.ExonPenalty} and \texttt{Repeat.UTRPenalty} penalties.

\paragraph{Post analyse}

No post analyse.

\paragraph{Graph}

Repeat intervals are visualized as grey blocks in the intergenic
track.




