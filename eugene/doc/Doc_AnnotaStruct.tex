% Documentation of the AnnotaStruct sensor

\subsection{\texttt{Sensor.AnnotaStruct}}

\paragraph{Description}

The sensor allows to seamlessly modify \EuGene\ underlying graph
weighting using a small langage that can directly modify the weights
of signals and contents edges in the graph. The plugin offers both
high-level entries and low-level entries in GFF format. 

The high-level entries allow to take into account information on:
\begin{itemize}
\item \textbf{transcribed sequences} (involving exons, introns, UTR,
  transcription start and transcription stop and splice sites) that
  may come from alignment of transcribed sequences (using spliced
  alignment algorithms such as sim4 or PASA).
\item \textbf{CDS} (involving exons, introns, translation start,
  translation stop and splice sites) that may come from other gene
  predictors that may predict CDS (either ab initio or homology based
  predictors).
\end{itemize}
The high-level entries are actually automatically expanded in
elementary (low-level) information as the plugin reads the data. The
way the expansion takes place is user-controllable through parameters.

Compared to the \texttt{Est} plgin, there is no data filtering
performed here which means that the plugin should rather be used on
consistent and fairly reliable data (eg. on existing gene predictions,
cDNA or EST cluster alignements rather than simple EST alignments that
would be better handled using the \texttt{Est} plugin).

The low-level entries allow to directly modify every edge of the
underlying prediction graph of \EuGene\ as the (now obsolete)
\texttt{User} plugin allowed. The weights of all signals edges
(transcription start and stop, translation start and stop, splice
sites, insertions and deletions) and contents edges (exons, introns,
UTR, UTR introns and intergenic regions) can be directly modified
using this plugin.

The sensor is activated by using the value TRUE for the parameter
\texttt{Sensor.AnnotaStruct.use} in the parameter file.

Here is an example of AnnotaStruct parameters definition :
\begin{Verbatim}[fontsize=\small]
AnnotaStruct.FileExtension
AnnotaStruct.Exon*
AnnotaStruct.Intron*
AnnotaStruct.CDS*
AnnotaStruct.StartType
AnnotaStruct.Start*
AnnotaStruct.StopType
AnnotaStruct.Stop*
AnnotaStruct.AccType
AnnotaStruct.Acc*
AnnotaStruct.DonType
AnnotaStruct.Don*
AnnotaStruct.TrStartType
AnnotaStruct.TrStart*
AnnotaStruct.TrStopType
AnnotaStruct.TrStop*
Sensor.AnnotaStruct.use  TRUE      # Use AnnotaStruct sensor
Sensor.AnnotaStruct      9         # Sensor priority
\end{Verbatim}

\paragraph{Input files format}

The plugin reads a GFF format file. Each line in this file forms an
elementary information which is directly interpreted by the plugin
independently of other lines. A GFF line is formed by a sequence of
separated fields: sequence name, source, feature, start, end, score,
strand and frame. The sequence name and source fields are ignored by
the plugin and can be set to user informative values.

Each line may either represent a high-level or a low-level
information.  Low-level informations use specific features for
specifying which signals and contents edges should be modified. For
signals, the following features are recognized:
\begin{itemize}
\item \texttt{trStart}: for transcription starts.
\item \texttt{trStop}: for transcription stops.
\item \texttt{start}: for translation starts (ATG).
\item \texttt{stop}: for translation stops.
\item \texttt{acc}: for acceptor splice sites.
\item \texttt{don}: for donor splice sites.
\item \texttt{ins}: for insertions (frameshift).
\item \texttt{del}: for deletion (frameshift).
\end{itemize}
In this case, the start and the strand field are used to indicate the
signal position. The score field is used to indicate the weight that
will be used to modify the existing weight. It is either a floating
point value between \texttt{-1e999} and \texttt{1e999} (that match
$-\infty$ and $\infty$ respectively in the format used) or a floating point
between $0.0$ and $1.0$ preceded by the letter $p$ (like probability).

\begin{enumerate}
\item In the first case, the score indicated is directly added to the
  weight of the signal edge (that corresponds to the fact that the
  signal is used). The other signal edge is unmodified.
\item In the second case, the score $s$ that appears fater the $p$ is
  treated as a (conditional) probability. The edge that corresponds to
  the fact that the signal is used receive a weight $\log(s)$ and the
  other edge $\log(1-p)$.
\end{enumerate}

For contents edges, the following features are recognized:
\begin{itemize}
\item \texttt{exon}: for coding exons.
\item \texttt{intron}: for introns separating coding exons.
\item \texttt{utr5}: for 5' UTR (untranslated terminal regions).
\item \texttt{utr3}: for 3' UTR.
\item \texttt{utr}: for both 5' or 3' UTR.
\item \texttt{intronutr}: for UTR introns.
\end{itemize}
The start and end fields together with the strand field delimit the
region considered. All corresponding contents edges will be modified
by the weight indicated in the score field.


High-level information may either be used to express knowledge about
potential \emph{transcribed sequences} or potential \emph{coding
  sequences}. For informations about CDS regions, the following
features may be used:
\begin{itemize}
\item \texttt{E.Init}: for an initial exon, this will automatically
  expand in the weight modification of a translation start and a donor
  site at the corresponding extremities on the indicated strand (using
  parameters \texttt{AnnotaStruct.Start*} and
  \texttt{AnnotaStruct.Don*} respectively as weights) and the contents
  modification for the exon in the corresponding frame and strand
  (using the score indicated in the \texttt{AnnotaStruct.CDS*}
  parameter).
\item \texttt{E.Intr}: for an intermediary exon, this will
  automatically expand in the weight modification of a donor and an
  acceptor site at the corresponding extremities on the indicated
  strand (using parameters \texttt{AnnotaStruct.Don*} and
  \texttt{AnnotaStruct.Acc*} respectively as weights) and the contents
  modification for the exon in the corresponding frame and strand
  (using the score indicated in the \texttt{AnnotaStruct.CDS*}
  parameter).
\item \texttt{E.Term}: for a terminal exon, this will automatically
  expand in the weight modification of an acceptor and a stop signal
  at the corresponding extremities on the indicated strand (using
  parameters \texttt{AnnotaStruct.Acc*} and \texttt{AnnotaStruct.Stop*}
  respectively as weights) and the contents modification for the exon
  in the corresponding frame and strand (using the score indicated in
  the \texttt{AnnotaStruct.CDS*} parameter).
\item \texttt{E.Sngl}: for a single exon gene, this will automatically
  expand in the weight modification of a translation start and stop
  signal at the corresponding extremities on the indicated strand
  (using parameters \texttt{AnnotaStruct.Start*} and
  \texttt{AnnotaStruct.Stop*} respectively as weights) and the
  contents modification for the exon in the corresponding frame and
  strand (using the score indicated in the \texttt{AnnotaStruct.CDS*}
  parameter).
\item \texttt{UTR5}, \texttt{UTR3}, \texttt{UTR}: although not part of
  the CDS, some gene predictors may predict UTR (non coding part of
  exons). These 3 features allow to inject this information by
  prespectively reweighting a transcription start, stop or both using
  the corresponding \texttt{AnnotaStruct.TrStart*},
  \texttt{AnnotaStruct.TrStop*} parameters and then by reweighting the
  UTR5, UTR3 or both contents edges (using the
  \texttt{AnnotaStruct.CDS*} parameter).
\item \texttt{Intron}: equivalent to \texttt{intron} except that the
  weight used comes from the \texttt{AnnotaStruct.CDS*} parameter). 
\end{itemize}

For information about transcribed sequences, the following features
are recognized:
\begin{itemize}
\item \texttt{E.Any}: any exon in the biological sense \textit{i.e.}
  either an exon or a UTR in the \EuGene\ sense. Frame is typically
  unknown (in this case, all coding frame in the indicated strand are
  considered). The corresponding contents region are modified
  accordingly to the \texttt{AnnotaStruct.Exon*} parameter.
\item \texttt{E.First}: the first biological exon (containing UTR and
  possibly part of CDS too)1. A transcription start signal is weighted
  according to the \texttt{AnnotaStruct.TrStart*} parameter value. The
  UTR5 and coding exon contents edge are reweighted according to the
  \texttt{AnnotaStruct.Exon*} parameter value.
\item \texttt{E.Last}: the last biological exon. A transcription stop
  signal is weighted according to the \texttt{AnnotaStruct.TrStop*}
  parameters value. The UTR3 and coding exon contents edge are
  reweighted according to the \texttt{AnnotaStruct.Exon*} parameter
  value.
\item \texttt{E.Extreme}: used for a biological exon on the extremity
  (either first or last).  A transcription start and stop are
  generated at each respective extremities according to the
  \texttt{AnnotaStruct.TrStart*} and \texttt{AnnotaStruct.TrStop*}
  parameter values. The UTR5, UTR3 and coding exon contents edges are
  reweighted according to the \texttt{AnnotaStruct.Exon*} parameter
  value.
\end{itemize}


Here is a high-level CDS based example:
\begin{Verbatim}[fontsize=\small]
ATSYNO FGENESH E.Init  3 33 0 + 3
ATSYNO FGENESH E.Term 45 75 0 + 3
\end{Verbatim}


\paragraph{Filtering input information}

No filtering beside syntax checking.

\paragraph{Integration of information}

The underlying graph edges are directly modified as indicated. 

\paragraph{Post analyse}

No post analyse.

\paragraph{Graph}

No plotting.




