% Documentation of the SMachine sensor

\subsubsection{\texttt{Sensor.SMachine}}


\paragraph{Description}

This plugin injects possible start and splice sites as predicted by
the SpliceMachine program. For more detail, see the publication
Degroeve, S., Saeys, Y., De Baets, B., Rouz�, P., Van de Peer, Y.
(2004) Predicting splice sites from high-dimensional local context
representations Bioinformatics.

There are two possible outputs for this program, the direct raw output
directly outputs SVM values (positive or negative real numbers with
arbitrary magnitude) or a rescaled output (fitting to a sigmoid) with
a positive ouput between 0 and 1 (with a probabilistic
interpretation). The two different outputs may lead to different
prediction performances in \EuGene.

The sensor is activated by setting the value 1 for the parameter \texttt{Sensor.SMachine.use}
in the parameter file. The score for start, acceptor and donor
prediction is rescaled by the parameters {\tt SMachine.startP*} and
{\tt SMachine.startB*} for starts, {\tt SMachine.accP*} and {\tt
  SMachine.accB*} for acceptors and {\tt SMachine.donP*} and {\tt
  SMachine.donB*} for donors (see below).  

The parameter {\tt SMachine.isScaled} indicates how the scores of
SpliceMachine are integrated in \EuGene\ (the details of the scaling
used in each case is given below. Note that this is the second rescaling
if the sigmoid fitting has been used in SpliceMachine).  The parameter
{\tt SMachine.cmd} contains the command which is launch if the
predictions files do not exist.

Here is an example of SMachine parameters definition.
\begin{Verbatim}[fontsize=\small]
SMachine.cmd            "splicemachine.pl "
SMachine.isScaled       1
SMachine.accP*          0.102032725565
SMachine.accB*          5.585
SMachine.donP*          0.020202707318
SMachine.donB*          27.670
SMachine.startP*        0.052
SMachine.startB*        0.308
Sensor.SMachine.use     1                # Use SMachine sensor
Sensor.SMachine         10               # Sensor priority
\end{Verbatim}

\paragraph{Native input files format}
The plugin reads the predictions of the program from two files whose
names are derived from the sequence name by adding the
\texttt{.spliceMSt} and \texttt{.spliceMAD} suffixes (respectively
prediction for the starts and splices sites)

The files with \texttt{.spliceMSt} and \texttt{.spliceMAD} suffixes are
obtained by running SpliceMachine which can be obtained at \\
\texttt{http://bioinformatics.psb.ugent.be/webtools/splicemachine/}

Here is an extract from a \texttt{.spliceMSt} file:
\begin{Verbatim}[fontsize=\small]
175 start_rev 0.024083
188 start 0.000151
195 start_rev 0.010081
261 start 0.001628
270 start 0.000026
[...]
\end{Verbatim}

Here is an extract from a \texttt{.spliceMAD} file:
\begin{Verbatim}[fontsize=\small]
210 acceptor_rev 0.066414
245 donor_rev 0.001345
628 acceptor 0.066414
1309 donor 0.000039
[...]
\end{Verbatim}

\paragraph{Gff3 input file format}
The gff3 input mode is activated by setting the value \texttt{GFF3} for the parameter
\texttt{SMachine.format} in the parameter file.
The plugin reads the predictions of the program from one file which
name is derived from the sequence name by adding the \texttt{.spliceM.gff3}

Accepted features (third column) :
\begin{itemize}
\item  SO:0000318 or start\_codon, the expected coordinates must correspond to the first nucleotide of the start codon.
\item  SO:0000164 or splice\_acceptor\_site, the expected coordinates must correspond to the AG/GT nucleotides.
\item  SO:0000163 or splice\_donor\_site, the expected coordinates must correspond to the AG/GT nucleotides.
\end{itemize}
If feature isn't these, line would be rejected. 

Here an extract of : seq14ac002535g4g5.tfa.spliceM.gff3
\begin{Verbatim}[fontsize=\tiny]
seq14	SMachine	start_codon	175	175	0.024083	-	.	ID=start_codon:seq14.1;
seq14	SMachine	start_codon	188	188	0.000151	+	.	ID=start_codon:seq14.2;
seq14	SMachine	acceptor_splice_site	210	211	0.066414	-	.	ID=acceptor_splice_site:seq14.1;
seq14	SMachine	donor_splice_site	244	245	0.001345	-	.	ID=donor_splice_site:seq14.1;
\end{Verbatim}


\paragraph{Filtering input information}

No filtering.

\paragraph{Integration of information}

The integrated score for start and donor/acceptor prediction is read
(columns 3).  The score read $s$ is rescaled using the {\tt
  SMachine.startP*} (\emph{P}) and {\tt SMachine.startB*} (\emph{B})
parameters for starts, {\tt SMachine.accP*} (\emph{P}) and {\tt
  SMachine.accB*} (\emph{B}) parameters for acceptors and {\tt
  SMachine.donP*} (\emph{P}) and {\tt SMachine.donB*} (\emph{B})
parameters for donors.


If ({\tt SMachine.isScaled} is set to {\tt 0}) then the rescaled score  $s'$ is:
\[s' = B*s-P\]
used when the signal is used. If the signal is not used, no penalty occurs.

If ({\tt SMachine.isScaled} is set to {\tt 1}) then the rescaled score  $s'$ is:
\[s' = B\log(s)-P\]
when the signal is used, and $\log(1.0-s^B*e^{-P})$ otherwise. 

If  ({\tt SMachine.isScaled} is set to {\tt 2}) then the rescaled score  $s'$ is
\[s' = B\log(s)-P\]
when the signal is used, nothing otherwise.


\paragraph{Post analyse}

No post analyse.

\paragraph{Graph}

Predicted starts are visible on exonix tracks as blue vertical lines
whose length indicates the site score.
Predicted splice sites are visible on the intronic tracks as green
(donor) and magenta (acceptor) vertical lines whose length indicates
the site score.




