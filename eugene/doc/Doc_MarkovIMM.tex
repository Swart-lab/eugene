% Documentation of the MarkovIMM sensor

\subsubsection{\texttt{Sensor.MarkovIMM}}

\paragraph{Description}

This plugin injects coding/intronic/utr/rna/intergenic likelihood as
modeled by interpolated Markov models (introduced in Glimmer, see S.
Salzberg, A. Delcher, S. Kasif, and O. White.{\em Microbial gene
  identification using interpolated Markov models} Nucleic Acids
Research 26:2 (1998), 544-548). These models are defined in a
so-called matrices file (located in the EUGENEDIR/models directory) whose name is indicated by the
\texttt{MarkovIMM.matname} parameter. Depending on the matrices file,
this may contain IMM for exons, introns and intergenic data and also
optionnally 5' and 3' UTR regions. If these 2 last IMMs are absent
from the matrices file, intronic models are used for UTR in eukaryote mode, and intergenic model is used for UTR in prokaryote mode.

The plugin is controlled by two further parameters:
\texttt{MarkovIMM.minGC} and \texttt{MarkovIMM.maxGC} which indicate
the GC scope of the matrices. If the GC\% of the sequence is out of
the scope, the plugin will give an equal null loglikelihood to all types
of regions.

By instanciating multiple \texttt{MarkovIMM} plugins (see
section~\ref{plug}), this enables the use of several IMM according to
the GC\% of the input sequence. 

The intergenic track can receive score in three possible ways
according to the \texttt{MarkovIMM.Interge\-nicModel} parameter value~:
\begin{enumerate}
\item[0] In this case, a $O^{th}$ order Markov model that is directly
  estimated from the frequencies in the sequence is used.
\item[1] The intergenic IMM matrix is directly used on the forward
  strand. This model will give a slightly different score to the
  sequence and to its reverse complement but is simple and works well
  on some organisms.
\item [2] The intergenic model is used on both strands and the mean of
  the two prababilities (forward and reverse) is used as the
  estimation. This is the default model to use. It has the advantage
  of giving the same score to a sequence and its reverse complement.
\end{enumerate}

The npcRNA track can receive score in two  possible ways
according to the \texttt{MarkovIMM.npcRNAModel} parameter value~:
\begin{enumerate}
\item  \texttt{gc} In this case, a $O^{th}$ order Markov model that is directly
  estimated from the frequencies in the sequence is used.
\item \texttt{numeric value} In this case, a constant score, which is the log of this value.
\end{enumerate}

The sensor is activated by either:
\begin{itemize}
\item the \texttt{-m} argument followed by the filename of the set of Markov models.
  \index{CmdFlags}{[MarkovIMM Markov models file] m}
\item the value 1 for the parameter \texttt{Sensor.MarkovIMM.use} in
  the parameter file.
\end{itemize}

Here is an example of MarkovIMM parameters definition.
\begin{Verbatim}[fontsize=\small]
MarkovIMM.matname    Ara2UTR.mat
MarkovIMM.minGC      0
MarkovIMM.maxGC      100
MarkovIMM.IntergenicModel  2
MarkovIMM.npcRNAModel  gc
MarkovIMM.maxOrder  8
Sensor.MarkovIMM.use 1
Sensor.MarkovIMM     1      # Sensor priority
\end{Verbatim}


\paragraph{Input files format}

No input files needed beyond the IMM matrix file. 

\paragraph{Integration of information}

On the forward intronic, exonic and UTR tracks, the probability that
the nucleotide at position $i$ appears given the nucleotides that
follow him is fetched in the IMM matrix file. The logarithm of this
probability is used as a penalty on the corresponding track.

For reverse tracks, the same process is used but the probability used
is the probability that the nucleotide at position $i$ appears given
the nucleotides that precede him.

For the intergenic track, the probability used is the mean of the
probability that the nucleotide at position $i$ appears given the
nucleotides that precede him and the probability that the nucleotide
at position $i$ appears given the nucleotides that follow him.  This
guarantees that a sequence and its reverse complement will receive the
same weights exactly.

\paragraph{Post analyse}

No post analyse.

\paragraph{Graph}

The likelihood of a subsequence of width \texttt{Output.window} is
computed for each IMM model and normalized over all these. The
corresponding normalized likelihood is plotted as a thin black line on
each track of the graphical output.


