% Documentation of the NG2 sensor

\subsubsection{\texttt{Sensor.NG2}}

\paragraph{Description}

This plugin injects possible splice sites as predicted by the
NetGene2 program. 

The sensor is activated by setting the value 1 for the parameter
\texttt{Sensor.NG2.use} in the parameter file. The score for acceptor
and donor prediction is rescaled by the parameters {\tt NG2.accP*} and
{\tt NG2.accB*} for acceptors and {\tt NG2.donP*} and {\tt NG2.donB*} for
donors (see below).

Here is an example of NG2 parameters definition.
\begin{Verbatim}[fontsize=\small]
NG2.accP*        0.903
NG2.accB*        5.585
NG2.donP*        0.980
NG2.donB*        27.670
Sensor.NG2.use   1
Sensor.NG2       1      # Sensor priority
\end{Verbatim}

\paragraph{Native input files format}
The plugin reads the prediction of the
program from two files whose names are derived from the sequence name
by adding the \texttt{.splices} and \texttt{.splicesR} suffixes
(respectively prediction for the forward and reverse strand).

The files with \texttt{.splices} and \texttt{.splicesR} suffixes are
obtained by running NetGene2 which can be obtained at
\texttt{http://www.cbs.dtu.dk/services/NetGene2/} and using the
detailed output of the software.

Here is an extract from \texttt{SYNO\_ARATH.fasta.splices}:
\begin{Verbatim}[fontsize=\small]
    396 C  0      0      0.903 2  0.835  0.862  0      0      -     - 
    397 T  0      0      0.858 3  0.828  0.869  0      0      -     - 
    398 C  0      0      0.873 1  0.822  0.876  0      0      -     - 
    399 A  0      0      0.826 2  0.816  0.882  0      0      -     - 
    400 G  0      0      0.869 3  0.809  0.889  0      0     -2.337 359 
    401 A  0      0      0.862 1  0.803  0.896  0      0      -     - 
    402 G  0      0      0.794 2  0.798  0.901  0      0     -2.337 359 
    403 C  0      0      0.809 3  0.792  0.907  0      0      -     - 
    404 A  0      0      0.833 1  0.786  0.914  0      0      -     - 
    405 G  0      0      0.823 2  0.780  0.920  0      0     -2.337 359 
    406 T  0      0      0.845 3  0.774  0.926  0      0      -     - 
    407 G  0      0      0.792 1  0.769  0.932  0      0      -     - 
    408 T  0      0      0.745 2  0.764  0.936  0      0      -     - 
    409 C  0      0      0.792 3  0.759  0.942  0      0      -     - 
    410 A  0      0      0.802 1  0.753  0.948  0      0      -     - 
    411 C  0      0      0.764 2  0.749  0.953  0      0      -     - 
[...]
\end{Verbatim}

To run Netgene2, the following parameters are used (for \emph{Arabidopsis
thaliana}): \texttt{netgene2 -a -e -p -r -s at <Fasta sequence>}.

\paragraph{Gff3 input file format}
The gff3 input mode is activated by setting the value \texttt{GFF3} for the parameter
\texttt{NG2.format} in the parameter file.
The plugin reads the predictions of the program from one file which
name is derived from the sequence name by adding the \texttt{.splices.gff3}.

Accepted features (third column) :
\begin{itemize}
\item  SO:0000164 or splice\_acceptor\_site ...
\item  SO:0000163 or splice\_donor\_site ...
\end{itemize}
If feature isn't those, line would be rejected. The expected coordinates must correspond to the AG/GT nucleotides.
Here an extract of : seq14ac002535g4g5.tfa.splices.gff3
\begin{Verbatim}[fontsize=\tiny]
seq14	NG2	SO:0000163	18	19	0.586	+	.	ID=SO:0000163:seq14.1;
seq14	NG2	SO:0000163	27	28	0.614	+	.	ID=SO:0000163:seq14.2;
seq14	NG2	SO:0000164	57	58	0.017	+	.	ID=SO:0000164:seq14.1;
\end{Verbatim}

\paragraph{Filtering input information}

No filtering.

\paragraph{Integration of information}

The integrated score for donor/acceptor prediction is read (columns 9
and 10). If it is not available (extremities of the sequence) then the
non integrated score is used (columns 3 and 4).

The score read $s$ is rescaled using the {\tt NG2.accP*} (\emph{P}) and {\tt
NG2.accB*} (\emph{B}) parameters for acceptors and {\tt NG2.donP*} (\emph{P}) and {\tt
NG2.donB*} (\emph{B}) parameters for donors as follows:

\[s' = B*s^P\]

All predictions that use a predicted splice site receive a $\log(s')$
penalty while those that go through a predicted splice site while they
could have used it receive a $\log(1-s')$ penalty.


\paragraph{Post analyse}

No post analyse.

\paragraph{Graph}

Predicted splice sites are visible on the intronic tracks as green
(donor) and magenta (acceptor) vertical lines whose length indicates
the site score.




