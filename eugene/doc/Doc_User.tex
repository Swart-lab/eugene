% Documentation of the User sensor

\subsection{\texttt{Sensor.User}}
\label{pluguser}

The user plugin is superseded by the AnnotaStruct\footnote{Also know
  as the BarbaTruc plugin.} plugin. It is conserved for compatibility
reasons.

\paragraph{Description}

The sensor allows to seamlessly modify \EuGene\ underlying graph
weighting using a small langage that can modify the weights of signals
and contents edges in the graph. The plugin priority is significant
here because it may directly set signal weights to a specific value
(to eg. remove a signal prediction). Contents edges are just penalized
(a weight is added to the existing weight).

The sensor is activated by either :
\begin{itemize}
\item the \texttt{-U} argument \index{CmdFlags}{[User activation] U}
\item the value TRUE for the parameter \texttt{Sensor.User.use} in the
  parameter file.
\end{itemize}
Here is an example of User parameters definition :
\begin{Verbatim}[fontsize=\small]
Sensor.User.use  TRUE      # Use User sensor
Sensor.USer      9         # Sensor priority
\end{Verbatim}

\paragraph{Input files format}

A small language is used to indicate which edges should have their
weight modified. Signal weight modification is indicated by the
keywords \texttt{start}, \texttt{stop}, \texttt{acceptor} and
\texttt{donor}. Each line should indicate the type of signal edge
which should be modified, the strand (either \texttt{f} or
\texttt{r}), the position and the weight $p$ to use for the signal.
$p$ can take any value between $0.0$ and $1.0$. The two signal edges
(corresponding respectivelyto the fact the the signal is used or not)
are respectively set to $\log(p)$ and $\log(1-p)$.

For contents edges, the keywords \texttt{exon}, \texttt{intron},
\texttt{utr3}, \texttt{utr5} and \texttt{intergenic} are used. Then
the strand is indicated (either \texttt{f} or \texttt{r}), then the
region where contents edge weights should be modified (as an interval
\texttt{[begin..end]}). Finally, the weight modification $p$ is given
as a floating point value. Any floating point value can be used
including two special values denoted \texttt{infinity} and
\texttt{-infinity} (with an obvious meaning).

Here is an example:
\begin{Verbatim}[fontsize=\small]
utr5 f [2..28] 10.0
acceptor f 2966 0.0
\end{Verbatim}

which delete any existing acceptor prediction at position 2966 on the
forward strand and favors UTR5 prediction on the 2-28 region.

\paragraph{Filtering input information}

No filtering.

\paragraph{Integration of information}

The underlying graph edges are directly modified as indicated. Pay
attention that signal edge weights are \emph{set} to the value
indicated while contents edge weights are just incremented by the
value indicated. The signal behavior is useful to be able to delete an
existing signal.

\paragraph{Post analyse}

No post analyse.

\paragraph{Graph}

No plotting.




