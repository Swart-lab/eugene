% Documentation of the NStart sensor

\subsubsection{\texttt{Sensor.NStart}}

\paragraph{Description}

This plugin injects possible translation starts as predicted by the
NetStart program. 

The sensor is activated by setting the value \texttt{1} for the parameter
\texttt{Sensor.NStart.use} in the parameter file. The score for acceptor
and donor prediction is rescaled by the parameters {\tt NStart.startP*} and
{\tt NStart.startB*} (see below).

Here is an example of NetStart parameters definition.
\begin{Verbatim}[fontsize=\small]
NStart.startP*     0.052
NStart.startB*     0.308
Sensor.NStart.use  1       # Use NStart sensor
Sensor.NStart      1       # Sensor priority
\end{Verbatim}

\paragraph{Native input files format}
The plugin reads the prediction of the
program from two files whose names are derived from the sequence name
by adding the \texttt{.starts} and \texttt{.startsR} suffixes
(respectively prediction for the forward and reverse strand).

The files with \texttt{.starts} and \texttt{.startsR} suffix are
obtained by running NetStart which can be obtained at
\texttt{http://www.cbs.dtu.dk/services/NetStart/} and using the
detailed output of the software.

Here is an extract from \texttt{SYNO\_ARATH.fasta.starts}:
\begin{Verbatim}[fontsize=\small]
  1089    0.256     -
  1146    0.214     -
  1251    0.618     Yes
  1299    0.197     -
  1474    0.526     Yes
  1535    0.112     -
  1559    0.490     -
  1638    0.401     -
  1674    0.569     Yes
  1678    0.147     -
  1740    0.299     -
  1752    0.187     -
[...]
\end{Verbatim}

To run NetStart, the following parameters are used (for \emph{Arabidopsis
thaliana}): \texttt{netstart -at <Fasta sequence>}.

\paragraph{Gff3 input file format}
The gff3 input mode is activated by setting the value \texttt{GFF3} for the parameter
\texttt{NStart.format} in the parameter file.
Then, the plugin reads the predictions of the program from one file which
name is derived from the sequence name by adding the \texttt{.nstart.gff3}

Accepted features (third column) :
\begin{itemize}
\item  SO:0000318 or start\_codon
\end{itemize}
If feature isn't these, line would be rejected. The expected coordinates must correspond to the first nucleotide of the start codon.

Here an extract of : seq14ac002535g4g5.tfa.starts.gff3
\begin{Verbatim}[fontsize=\tiny]
seq14	NStart	SO:0000318	64	64	0.299	+	.	ID=SO:0000318:seq14.0;
seq14	NStart	SO:0000318	74	74	0.249	+	.	ID=SO:0000318:seq14.1;
\end{Verbatim}


\paragraph{Filtering input information}

No filtering.

\paragraph{Integration of information}

The integrated score for start prediction is read (column 2).  The
score read $s$ is rescaled using the {\tt NStart.startP*} (\emph{P}) and {\tt
NStart.startB*} (\emph{B}) as follows:

\[s' = e^{-P}*s^B\]

All predictions that use a predicted start receive a $\log(s')$
penalty while those that go through a predicted start while they
could have used it receive a $\log(1-s')$ penalty.


\paragraph{Post analyse}

No post analyse.

\paragraph{Graph}

Predicted starts are visible on exonix tracks as blue vertical lines
whose length indicates the site score.





