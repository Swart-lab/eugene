% Documentation of the GCPlot sensor

\subsection{\texttt{Sensor.GCPlot}}

\paragraph{Description}

The GCPlot sensor allows to add to the graphical representation a plot
of basic composition statistics on the sequence. The sensor is
activated by setting the parameter \texttt{Sensor.GCPlot.use} to
\texttt{TRUE} in the parameter file. The composition statistics
represented can be arbitrarily chosen. For example, the
GC\%=$\frac{G+C}{A+T+G+C}$ is selected by setting \texttt{GCPlot.up}
to \texttt{GC} and \texttt{GCPlot.over} to \texttt{ATGC}. Statistics
on the 3rd base of each codon are automatically computed and plotted.

The color (integer between 0 and 8),, the smoothing window width and a
zooming factor can be specified. The zooming factor for the 3rd base
in each codon in zoomed using specific zooming factor
\texttt{GCPlot.Zoom}

Here is an example of a GCPlot parameter definition :
\begin{Verbatim}[fontsize=\small]
Sensor.GCPlot.use  TRUE      # use GCPlot sensor
Sensor.GCPlot      10        # sensor priority
GCPlot.Up  GC
GCPlot.Over  ATGC
GCPlot.Smooth 98
GCPlot.Color  5    #light green
GCPlot.Zoom 2.0
GCPlot.Zoom3 1.0
\end{Verbatim}

\paragraph{Input files format}

No input file.

\paragraph{Filtering input information}

No filter.

\paragraph{Integration of information}

This sensor does not influence prediction.

\paragraph{Post analyse}

No post analyse.

\paragraph{Graph}

The composition statistics is plotted on the intergenic (IG) track.
The same statistics computed on the 3rd position of each codon is
plotted on the 6 exonic tracks.

