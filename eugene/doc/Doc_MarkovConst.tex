% Documentation of the MarkovConst sensor

\subsubsection{\texttt{Sensor.MarkovConst}}

\paragraph{Description}

A simulated content sensor that gives constant probabilities (as
indicated in the \texttt{MarkovConst\-.Coding*},
\texttt{MarkovConst.Intron*}, \texttt{MarkovConst.IntronUTR*},
\texttt{MarkovConst.UTR5*}, \texttt{Mark\-ovConst.UTR3*},
\texttt{MarkovConst.UIR*} and \texttt{MarkovConst.Inter*} parameters).
to all positions for each region type. As the MarkovIMM sensor, the
plugin is controlled by two further parameters:
\texttt{MarkovConst.minGC} and \texttt{MarkovConst.maxGC} which
indicate the GC scope of the contents sensor. If the GC\% of the
sequence is out of the scope, the plugin will give an equal null
loglikelihood to all types of regions.

Used for testing purposes and for simulating the exponential length
distributions of HMM.

Here is an example of MarkovConst parameters definition.
\begin{Verbatim}[fontsize=\small]
MarkovConst.Coding*        1.0
MarkovConst.Intron*        1.0
MarkovConst.IntronUTR*     0.98
MarkovConst.UTR5*          0.99
MarkovConst.UTR3*          0.99
MarkovConst.UIR*           0.99
MarkovConst.RNA*           0.99
MarkovConst.Inter*         1.0
MarkovConst.affectedStrand 0
MarkovConst.minGC[0]       0
MarkovConst.maxGC[0]       100
Sensor.MarkovConst.use     1   # Use MarkovConst sensor
Sensor.MarkovConst         1      # Sensor priority
\end{Verbatim}

\paragraph{Input files format}

No input files needed.

\paragraph{Integration of information}

If the GC\% of the sequence handled is between
\texttt{MarkovConst.minGC} and \texttt{MarkovConst.maxGC} then in
every position, in all possible states, the prediction is penalized by
the logarithm of the corresponding parameter:
\texttt{MarkovConst.Coding*}, \texttt{MarkovConst.Intron*},
\texttt{MarkovConst.IntronUTR*}, \texttt{MarkovConst.UTR5*},
\texttt{MarkovConst.UTR3*}, \texttt{MarkovConst.UIR*}, 
\texttt{MarkovConst.RNA*} and \texttt{MarkovConst.Inter*}.

The \texttt{MarkovConst.affectedStrand} parameter allows to specify the affected strand and thus to turn off the prediction on the opposite strand.
To predict only on forward strand set \texttt{MarkovConst.affectedStrand} parameter value to 1;
to predict only one reverse one set it to -1, and to predict on both strands let the value to 0.

\paragraph{Post analyse}

No post analyse.

\paragraph{Graph}

No plotting.

