% Documentation of the SPred sensor

\subsubsection{\texttt{Sensor.SPred}}

\paragraph{Description}

This plugin injects possible splice sites as predicted by the
SplicePredictor program. 

The sensor is activated by setting the value 1 for the parameter
\texttt{Sensor.SPred.use} in the parameter file. The score for acceptor
and donor prediction is rescaled by the parameters {\tt SPred.accP*} and
{\tt SPred.accB*} for acceptors and {\tt SPred.donP*} and {\tt SPred.donB*} for
donors (see below).

Here is an example of SPred parameters definition.
\begin{Verbatim}[fontsize=\small]
SPred.accP*        0.987
SPred.accB*        3.850
SPred.donP*        0.929
SPred.donB*        10.800
Sensor.SPred.use   1
Sensor.SPred       1      # Sensor priority
\end{Verbatim}

\paragraph{Native input files format}
The plugin reads the prediction of
the program from two files whose names are derived from the sequence
name by adding the \texttt{.spliceP} and \texttt{.splicePR} suffixes
(respectively prediction for the forward and reverse strand).

The files with \texttt{.spliceP} and \texttt{.splicePR} suffix are
obtained by running SplicePredictor which can be obtained at
\texttt{http://bioinformatics.iastate.edu/cgi-bin/sp.cgi}.

Here is an extract from \texttt{SYNO\_ARATH.fasta.spliceP}:
\begin{Verbatim}[fontsize=\small]
[...]
A     <- 568 tgacttcggatgcAGaa   0.006 0.000 0.000  3 (1 1 1) IAEEEDA-E-EEEDIAD
A     <- 571 cttcggatgcagaAGgg   0.001 0.000 0.000  3 (1 1 1) AEEEDAE-E-EEDIADI
D -->    573           aggGTatga 0.176 0.009 0.000  6 (2 3 1) EEEDAEE-E-EDIADII
A     <- 582 gaagggtatgatcAGgt   0.003 0.000 0.000  3 (1 1 1) EEDAEEE-E-DIADIII
D -----> 583           cagGTaatt 0.964 0.256 0.788 15 (5 5 5) EDAEEEE-D-IAEEEED
A     <- 658 tatgataatccttAGac   0.001 0.000 0.000  3 (1 1 1) DAEEEED-I-AEEEEDI
A   <--- 698 ttgggtggataatAGgt   0.273 0.049 0.266 10 (3 3 4) AEEEEDI-A-EEEEDII
D ->     699           tagGTagaa 0.006 0.000 0.000  3 (1 1 1) AEEEDIA-E-EEEDIII
A     <- 702 gtggataataggtAGaa   0.001 0.000 0.000  3 (1 1 1) IIADIAD-I-IIIIIAD
D ->     709           ctgGTtcga 0.005 0.000 0.000  3 (1 1 1) IADIADI-I-IIIIAED
A     <- 744 cagtatctgtacaAGgt   0.007 0.000 0.000  3 (1 1 1) ADIADII-I-IIIAEDI
D ->     745           aagGTacta 0.042 0.000 0.000  3 (1 1 1) DIADIII-I-IIAEDIA
A     <- 756 aaggtactattgtAGct   0.007 0.000 0.000  3 (1 1 1) IADIIII-I-IAEDIIA
A     <- 760 tactattgtagctAGcc   0.002 0.000 0.000  3 (1 1 1) ADIIIII-I-AEDIIAE
A     <- 764 attgtagctagccAGgg   0.025 0.000 0.023  4 (1 1 2) DIIIIII-A-EDIIAEE
D ->     792           aagGTggag 0.057 0.001 0.000  3 (1 1 1) IIIIIIA-E-DIIAEEE
[...]
\end{Verbatim}

\paragraph{Gff3 input file format}

The gff3 input mode is activated by setting the value \texttt{GFF3}
for the parameter \texttt{SPred.format} in the parameter file.  The
plugin reads the predictions of the program from one file which name
is derived from the sequence name by adding the \texttt{.spliceP.gff3}
extension.

Accepted features (third column):\\
\begin{itemize}
\item  SO:0000164 or splice\_acceptor\_site 
\item  SO:0000163 or splice\_donor\_site 
\end{itemize}
If the feature used isn't one of those, the line will be rejected. The
expected coordinates must match the AG/GT nucleotides.

Here an extract of \texttt{seq14ac002535g4g5.tfa.spliceP.gff3}.
\begin{Verbatim}[fontsize=\tiny]
seq14   SPred   SO:0000164      922     923     0.002   +       .       ID=SO:0000164:seq14.16;
seq14   SPred   SO:0000163      1098    1099    0.137   +       .       ID=SO:0000163:seq14.8;
\end{Verbatim}

\paragraph{Filtering input information}

No filtering.

\paragraph{Integration of information}

One of the SplicePredictor scores $s$ for a given position is
rescaled using the $\log(\alpha.s^\beta)$ function. The four parameters
\texttt{SPred.accP*}, \texttt{SPred.accB*}, \texttt{SPred.donP*},
\texttt{SPred.donB*} indicates the values of these $\alpha$ and $\beta$
parameters for acceptor sites and donor sites respectively.  These
parameters have been estimated on existing data.

\paragraph{Post analyse}

No post analyse.

\paragraph{Graph}

Predicted splice sites are visible on the intronic tracks as green
(donor) and magenta (acceptor) vertical lines whose length indicates
the site score.
