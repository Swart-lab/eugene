% Documentation of the GSplicer sensor

\subsubsection{\texttt{Sensor.GSplicer}}

\paragraph{Description}

The GSplicer sensor injects possible splice sites as predicted by the
GeneSplicer program. The sensor is
activated by the value 1 for the parameter
\texttt{Sensor.GSplicer.use} in the parameter file.  Here is an
example of GSplicer parameters definition :
\begin{Verbatim}[fontsize=\small]
GSplicer.coefAcc*    0.8300      #
GSplicer.penAcc*     7.7000      # GSplicer parameters (rescaling)
GSplicer.coefDon*    1.3153      # See the Integration of
GSplicer.penDon*     10.1600     # information section.
Sensor.GSplicer.use  1        # Use GSplicer sensor
Sensor.GSplicer      1           # Sensor priority
\end{Verbatim}

\paragraph{Native input files format}
The plugin reads the prediction of the
program from one file whose name is derived from the sequence name by
adding the \texttt{.Gsplicer} suffix. This file describes the predicted
splice sites for the forward and reverse strand.

The files \texttt{.Gsplicer} describe the predicted splice sites
sorted by position on the sequence (as given by GeneSplicer).
The format of a line is : \texttt{<End5> <End3> <Score> <confidence>
  <splice\_site\_type>}

Here is an extract from \texttt{SYNO\_ARATH.fasta.Gsplicer} :
\begin{Verbatim}[fontsize=\small]
422 423 17.275659 High donor
512 513 15.534963 High acceptor
583 584 9.516534 Medium donor
697 698 6.432014 Medium acceptor
745 746 2.028095 Medium donor
810 811 9.255683 Medium donor
896 897 9.772425 Medium acceptor
1019 1020 10.580889 Medium donor
1105 1106 5.318046 Medium acceptor
1177 1178 7.345249 Medium acceptor
1203 1204 3.989773 Medium donor
1269 1270 15.249301 High acceptor
[...]
\end{Verbatim}

These files can be obtained by launching the GeneSplicer software
available at:
\begin{itemize}
\item  web site : \texttt{http://www.tigr.org/tdb/GeneSplicer/index.shtml}.
\item  ftp serveur : \texttt{ftp://ftp.tigr.org/pub/software/GeneSplicer}.
\end{itemize}

GeneSplicer is launched with the command:
\begin{Verbatim}[fontsize=\small]
genesplicer SEQ_FASTA GENOME_TRAINING_DIRECTORY [options] > SEQ_FASTA.Gsplicer
\end{Verbatim}
where options are:
\begin{itemize}
\item $-a$ $t$ : Choose $t$ as a threshold for the acceptor sites
\item $-d$ $t$ : Choose $t$ as a threshold for the donor sites
\item $-e$ $n$ : The maximum acceptor score within $n$ bp is chosen
\item $-i$ $n$ : The maximum donor score within $n$ bp is chosen
\end{itemize}

\paragraph{Gff3 input file format}

The gff3 input mode is activated by setting the value \texttt{GFF3}
for the parameter \texttt{Gsplicer.format} in the parameter file.  The
plugin reads the predictions of the program from one file which name
is derived from the sequence name by adding the
\texttt{.Gsplicer.gff3} extension.

Accepted features (third column):\\
\begin{itemize}
\item  SO:0000164 or splice\_acceptor\_site 
\item  SO:0000163 or splice\_donor\_site 
\end{itemize}
If the feature used isn't one of those, the line will be rejected. The
expected coordinates must correspond to the AG/GT nucleotides (this is
checked).  Here an extract of \texttt{seq14ac002535g4g5.tfa.Gsplicer.gff3}
\begin{Verbatim}[fontsize=\tiny]
seq14	GSplicer	SO:0000164	210	211	9.013469	-	.	ID=SO:0000164:seq14.1
seq14	GSplicer	SO:0000163	244	245	3.350242	-	.	ID=SO:0000163:seq14.1
seq14	GSplicer	SO:0000163	307	308	4.115466	-	.	ID=SO:0000163:seq14.2
\end{Verbatim}

\paragraph{Filtering input information}

No filter.

\paragraph{Integration of information}

The procedure consists in weigthing the graph used by \EuGene. For
each predicted site the edge corresponding to the good transition is
weighted by: $(s_i * coef) - pen$.  Where $s_i$ is the score given by
GeneSplicer, $coef$ and $pen$ are given in the parameter
file.

\shrink{
A set of 8 vectors is used. The vectors are:
\begin{itemize}
\item $vPosAccF$ the forward acceptor predicted positions
\item $vValAccF$ the forward acceptor score at position $vPosAccF$
\item $vPosAccR$ the reverse acceptor predicted positions
\item $vValAccR$ the reverse acceptor score at position $vPosAccR$
\item $vPosDonF$ the forward donor predicted positions
\item $vValDonF$ the forward donor score at position $vPosDonF$
\item $vPosDonR$ the reverse donor predicted positions
\item $vValDonR$ the reverse donor score at position $vPosDonR$
\end{itemize}

For each position if one of the 4 ``position vectors'' contains the
query position:
\begin{itemize}
\item for forward and reverse acceptor sites : $(vValAcc * coefAcc) -
  penAcc$ is added to the corresponding transition (intron track to exon
  track according to the phase)
\item for forward and reverse donor sites : $(vValDon * coefDon) -
  penDon$ is added to the corresponding transition (exon track to intron
  track according to the phase)
\end{itemize}
}

\paragraph{Post analyse}

No post analyse.

\paragraph{Graph}

Predicted splice sites are visible on the intronic tracks as green
(donor) and magenta (acceptor) vertical lines whose length indicates
the site score.

